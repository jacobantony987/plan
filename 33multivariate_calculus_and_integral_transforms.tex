%Text Books : \cite{apostol}, \cite{rudin}

%Module 1:
%The Weirstrass theorem, other forms of Fourier series, the Fourier integral theorem, the exponential form of the Fourier integral theorem, integral transforms and convolutions, the convolution theorem for Fourier transforms.
%(Chapter 11 Sections 11.15 to 11.21 of \cite{apostol}) (20 hours.)
%Module 2:
%Multivariable Differential Calculus, The directional derivative, directional derivatives and continuity, the total derivative, the total derivative expressed in terms of partial derivatives, An application of complex- valued functions, the matrix of a linear function, the Jacobian matrix, the matrix form of the chain rule. Implicit functions and extremum problems, the mean value theorem for differentiable functions,
%(Chapter 12 Sections. 12.1 to 12.11 of \cite{apostol}) (22 hours.)
%Module 3: 
%A sufficient condition for differentiability, a sufficient condition for equality of mixed partial derivatives, functions with non-zero Jacobian determinant, the inverse function theorem ,the implicit function theorem, extrema of real- valued functions of one variable, extrema of real-valued functions of several variables.
%Chapter 12 Sections-. 12.12 to 12.13 of \cite{apostol} 
%Chapter 13 Sections-. 13.1 to 13.6 of \cite{apostol} (28 hours.)
%Module 4:
%Integration of Differential Forms- Integration, primitive mappings, partitions of unity, change of variables, differential forms.
%(Chapter 10 Sections. 10.1 to 10.14 of \cite{rudin}) (20 hours)

\chapter{ME010303 Multivariate Calculus \& Integral Transforms}
\textit{Textbooks : Tom M. Apostol, Mathematical Analysis, 2nd Edition, Addison-Wesley, 1974\\Walter Rudin, Principles of mathematical analysis, 3rd Edition}
\begin{enumerate}[label=Week \arabic*]
	\item Weierstrass approximation theorem, other forms of Fourier series, Fourier integral theorem, exponential form of Fourier integral theorem, integral transforms. Reading : \S 11.15-20
	\item directional derivatives, total derivative, complex valued functions, matrix of linear functions, Jacobian matrix. Reading: \S 12.1-8
	\item chain rule, matrix form of chain rule, mean-value theorem. Reading : \S 12.9-11
	\item convolution theorem for Fourier transforms. Reading : \S 11.21 (pending)\\
		\textbf{Oct 12, 2020} Internal Examination Module 1 \& 2
	\item sufficient condition for differentiability, sufficient condition for equality of partial derivatives. Reading : \S 12.12-13\\
		\textbf{Oct 16,2020}
	\item implicit function, Jacobian determinant $J_f(\overline{x})$, Jacobian determinant of complex-valued functions, properties of functions with non-zero Jacobian determinant, inverse function therorem, implicit function theorem. Reading : \S 13.1-4\\
		\textbf{Oct 23, 2020}
	\item extrema of function on one variable, extrema of functions on several variables. Reading : \S 13:5-6\\
		\textbf{Oct 30, 2020}
	\item k-cell $I_k$, integration over k-cell, support, primitive mappings, flip, local represenation as composition of primitives and flips, partitions of unity, change of variables on continuous functions with compact support.  Reading : \S 10.1-9\\
		\textbf{Nov 6, 2020}
	\item k-surface, k-form (differential form of order $k$), properties of $k$-forms, basic $k$-forms. Reading : \S 10.10-14\\
		\textbf{Nov 13, 2020}
\end{enumerate}
