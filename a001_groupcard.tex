\documentclass[a6paper,10pt]{book}
\usepackage{glimpse}
\usepackage[width=148mm,height=105mm]{geometry}
\pagestyle{empty}
\begin{document}

\begin{center}\textbf{$(Z_7,+_7)$}\end{center}
\[ Z_7 = \{ 0,\ 1,\ 2,\ 3,\ 4,\ 5,\ 6 \} \]
\[ \text{Week} = \{ Sun, Mon, Tue, Wed, Thu, Fri, Sat \} \]

\begin{figure}[h]
	\centering
	\begin{tabular}{c|ccccccc}
		$\ast$ & 0 & 1 & 2 & 3 & 4 & 5 & 6 \\ \hline
		0 & 0 & 1 & 2 & 3 & 4 & 5 & 6 \\
		1 & 1 & 2 & 3 & 4 & 5 & 6 & 0 \\
		2 & 2 & 3 & 4 & 5 & 6 & 0 & 1 \\
		3 & 3 & 4 & 5 & 6 & 0 & 1 & 2 \\
		4 & 4 & 5 & 6 & 0 & 1 & 2 & 3 \\
		5 & 5 & 6 & 0 & 1 & 2 & 3 & 4 \\
		6 & 6 & 0 & 1 & 2 & 3 & 4 & 5
	\end{tabular}
\end{figure}

\begin{center}
	\textit{Four days after a friday is a tuesday.}\\
	For example : $ 5 \ast 4 = 5 +_7 4 = 9 \pmod{7} = 2$
\end{center}
\pagebreak

\begin{center} \textbf{$(Z_7^\ast,\times_7)$} \end{center}
\[ Z_7 = \{ 1,\ 2,\ 3,\ 4,\ 5,\ 6 \} \]
\begin{figure}[h]
	\centering
	\begin{tabular}{c|cccccc}
		$\ast$ & 1 & 2 & 3 & 4 & 5 & 6 \\ \hline
		1 & 1 & 2 & 3 & 4 & 5 & 6 \\
		2 & 2 & 4 & 6 & 1 & 3 & 5 \\
		3 & 3 & 6 & 2 & 5 & 1 & 4 \\
		4 & 4 & 1 & 5 & 2 & 6 & 3 \\
		5 & 5 & 3 & 1 & 6 & 4 & 2 \\
		6 & 6 & 5 & 4 & 3 & 2 & 1
	\end{tabular}
\end{figure}

\begin{center}
	For example : $ 5 \ast 4 = 5 \times_7 4 = 20 \pmod{7} = 6$
\end{center}
\pagebreak

\begin{center}\textbf{Symmetric Group, $S_3$}\end{center}
\[S_3 = \{ (),\ (1\ 2),\ (1\ 3),\ (2\ 3),\ (1\ 2\ 3),\ (1\ 3\ 2) \}\]

\begin{figure}[h]
	\centering
	\scalebox{0.7}{
	\begin{tabular}{c|cccccc}
		$\ast$ & () & (1 2) & (1 3) & (2 3) & (1 2 3) & (1 3 2) \\ \hline
		() & () & (1 2) & (1 3) & (2 3) & (1 2 3) & (1 3 2) \\
		(1 2) & (1 2) & () & (1 2 3) & (1 3 2) & (1 3) & (2 3) \\
		(1 3) & (1 3) & (1 3 2) & () & (1 2 3) & (2 3) & (1 2) \\
		(2 3) & (2 3) & (1 2 3) & (1 3 2) & () & (1 2) & (1 3) \\
		(1 2 3) & (1 2 3) & (2 3) & (1 2) & (1 3) & (1 3 2) & () \\
		(1 3 2) & (1 3 2) & (1 3) & (2 3) & (1 2) & () & (1 2 3)
	\end{tabular}}
\end{figure}

\begin{center}	For example : $(1\ 3) \ast (1\ 2) = (1\ 3\ 2)$ \end{center}
\pagebreak

\begin{center} \textbf{Dihedral Group, $D_8$}\end{center}
\[ D_8 = \{ e,\sigma,\sigma^2,\sigma^3,\mu,\mu\sigma,\mu\sigma^2,\mu\sigma^3 \} \]

\begin{figure}[h]
	\centering
	\scalebox{0.8}{
	\begin{tabular}{c|cccccccc}
		$\ast$ & $e$ & $\sigma$ & $\sigma^2$ & $\sigma^3$ & $\mu$ & $\mu\sigma$ & $\mu\sigma^2$ & $\mu\sigma^3$  \\ \hline
		$e$ & $e$ & $\sigma$ & $\sigma^2$ & $\sigma^3$ & $\mu$ & $\mu\sigma$ & $\mu\sigma^2$ & $\mu\sigma^3$  \\
		$\sigma$ & $\sigma$ & $\sigma^2$ & $\sigma^3$ & $e$ & $\mu\sigma^3$ & $\mu$ & $\mu\sigma$ & $\mu\sigma^2$  \\
		$\sigma^2$ & $\sigma^2$ & $\sigma^3$ & $e$ & $\sigma$ & $\mu\sigma^2$ & $\mu\sigma^3$ & $\mu$ & $\mu\sigma$  \\
		$\sigma^3$ & $\sigma^3$ & $e$ & $\sigma$ & $\sigma^2$ & $\mu\sigma$ & $\mu\sigma^2$ & $\mu\sigma^3$ & $\mu$  \\
		$\mu$ & $\mu$ & $\mu\sigma$ & $\mu\sigma^2$ & $\mu\sigma^3$ & $e$ & $\sigma$ & $\sigma^2$ & $\sigma^3$  \\
		$\mu\sigma$ & $\mu\sigma$ & $\mu\sigma^2$ & $\mu\sigma^3$ & $\mu$ & $\sigma^3$ & $e$ & $\sigma$ & $\sigma^2$  \\
		$\mu\sigma^2$ & $\mu\sigma^2$ & $\mu\sigma^3$ & $\mu$ & $\mu\sigma$ & $\sigma^2$ & $\sigma^3$ & $e$ & $\sigma$  \\
		$\mu\sigma^3$ & $\mu\sigma^3$ & $\mu$ & $\mu\sigma$ & $\mu\sigma^2$ & $\sigma$ & $\sigma^2$ & $\sigma^3$ & $e$  \\
	\end{tabular}}
\end{figure}

\begin{center}
	\textit{(Hint : $\sigma\mu = \mu\sigma^3$)}\\
	For example : $\mu\sigma \ast \mu\sigma^2 = \mu(\sigma\mu)\sigma^2 = \sigma$
\end{center}

\end{document}
