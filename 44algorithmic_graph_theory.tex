%Text Books : \cite{chartrand}
%Module 1 : Introduction to Graphs and Algorithms
%What is graph? The degree of a vertex, isomorphic graphs, subgraphs, degree sequences, connected graphs, cutvertices and blocks, special graphs, digraphs, algorithmic complexity, Search algorithms, sorting algorithms, greedy algorithms,, representing graphs in a computer.
%( Chapter 1 Sections 1.1 to 1.9, Chapter 2 Sections 2.1, 2.2 , 2.3, 2.5 and 2.6 of \cite{chartrand} ) (24 hours)
%Module 2 : Trees, paths and distances
%Properties of trees, rooted trees, Depth-first search, breadth-first search, the minimum spanning tree problem
%Distance in a graphs, distance in weighted graphs, the centre and median of a graph, Activity digraphs and critical paths.
%(Chapter 3 sections 3.1 to 3.3, 3.4 and 3.5 , Chapter 4 of \cite{chartrand} ) (22 hours)
%Module 3 : Networks
%An introduction to networks, the max-flow min-cut theorem, the max-flow min-cut algorithm, Connectivity and edge connectivity, Mengers theorem.
%( Chapter 5 sections 5.1 , 5.2 , 5.3 and 5.5 of \cite{chartrand} ) (22 hours)
%Module 4 : Matchings and Factorizations
%An introduction to matchings, maximum matchings in a bipartite graph, Factorizations, Block Designs.
%(Chapter 6 sections 6.1 , 6.2 , 6.4 and 6.5 of \cite{chartrand} ) (22 hours)

\chapter{ME800402 Algorithmic Graph Theory}
\begin{enumerate}[label=Week \arabic*]
	\item Reading \S1.1-1.9
	\begin{enumerate}[label=Day \arabic*]
		\item Graph, Degree of Vertex, Graph Isomorphism, Subgraph, Degree Sequence\\
			Graphic Degree Sequence : \textbf{Havel-Hakini} \\
 			Reading : \S1.1-1.5
		\item Connected Graph, Cut-vertex, Bridge, Block, Special Graphs: Complete, $n$ -partite, and HyperCube, Digraph, Indegree \& Outdegree (id,od), Semiwalk, Weakly connected, Symmetric, Tournament, Multidigraph, Pseudodigraph.\\
			Every $u-v$ walk contains a $u-v$ path.\\
			Bridge Characterisation : bridge won't belong to any cycle\\
			Bipartite Characterisation : No odd cycles \\
			If there is a Cut-vertex, then thereare two end-blocks.\\
			Reading : \S1.6-1.9
		\item 
		\item 
		\item 
	\end{enumerate}
	\item Reading
	\begin{enumerate}[label=Day \arabic*]
		\item 
		\item 
		\item 
		\item 
		\item 
	\end{enumerate}
	\item Reading
	\begin{enumerate}[label=Day \arabic*]
		\item 
		\item 
		\item 
		\item 
		\item 
	\end{enumerate}
	\item Reading
	\begin{enumerate}[label=Day \arabic*]
		\item 
		\item 
		\item 
		\item 
		\item 
	\end{enumerate}
	\item Reading
	\begin{enumerate}[label=Day \arabic*]
		\item 
		\item 
		\item 
		\item 
		\item 
	\end{enumerate}
\end{enumerate}
